\documentclass[a4paper, twoside, english]{article}

\usepackage{amsmath}
\usepackage{amsfonts}
\usepackage{ihci}
\usepackage{graphicx}
\usepackage{subfig}

\graphicspath{{./../figures/}}

\title{Exercise 1 \\ 3D Computer Vision}  % Replace "Template Report" with Exercise 1, Exercise 2, etc
\author{Albert Garaev, Ksenia Novikova, Mukhammadsodik Khabibulloev }    % Replace with your names
\date{09.11.2021}                              % Replace with current date

\begin{document}

\maketitle

\section{Introduction}

This report presents the results of the theoretical and practical parts of Exercise 1.

\section{Theory}
\subsection{ Properties of Rotation Matrices}
\textit{(a) The task is showing the fact that the rows and columns of R are orthonomal\\ (orthogonal and of length 1).}\\
Let's begin with the rotation matrix around the x-axis.\\
Rotation by $\phi$ aroud $x$ axis:
\begin{equation}
	R_x = \left[
	\begin{array}{ccc}
		1 & 0 & 0 \\
		0 & cos \phi & -sin \phi \\
		0 & sin \phi & cos \phi
	\end{array}
	\right].
	\label{eq:kmatrixX}
\end{equation}
We can prove that the rows and columns of $R$ are orthogonal by proving the equations $R^{-1} = R^T$ and $det(R)=1.$
Let's calculate ${{R_x}^{-1}}={{(R_x^{*})^T}\over{det(R_x)}}$, we need to calculate $det(R_x)$ using minors M.

${det(R_x)=(using\ first\ row)=1*(cos \phi * cos \phi - (-sin \phi * sin \phi)) - 0 + 0 = \cos^2 \phi + \sin^2 \phi = 1}$\\
Find transposed matrix $R_x^T$:
\begin{equation*}
	{R_x^T} = \left[
	\begin{array}{ccc}
		1 & 0 & 0 \\
		0 & cos \phi & sin \phi \\
		0 & -sin \phi & cos \phi
	\end{array}
	\right]
	\label{eq:kmatrixXT}
\end{equation*}
Find minor matrix M:
\begin{equation*}
	M=\left[
	\begin{array}{ccc}
		1&0&0\\
		0&\cos\phi&\sin\phi\\
		0&-\sin\phi&\cos\phi
	\end{array}
	\right]
\end{equation*}
Find adjugate matrix $R_x^{*}$:
\begin{equation*}
R_x^{*}=	\left[
	\begin{array}{ccc}
		1&0&0\\
		0&\cos\phi&-\sin\phi\\
		0&\sin\phi&\cos\phi
	\end{array}
	\right]
\end{equation*}
Thus, we can find the value of ${R_x}^{-1}$
\begin{equation*}
{{R_x}^{-1}}={{(R_x^{*})^T}\over{det(R_x)}}= {{\left[ \begin{array}{ccc}
	1 & 0 & 0 \\
	0 & cos \phi & sin \phi \\
	0 & -sin \phi & cos \phi
\end{array} \right]}\over {1}} =  {\left[ \begin{array}{ccc}
1 & 0 & 0 \\
0 & cos \phi & sin \phi \\
0 & -sin \phi & cos \phi
\end{array} \right]} = R^T
\end{equation*} Q.E.D.\\
Now we will prove that the length of any row or column is equal to 1.\\
 {Row 1 $r_1=(1,0,0)$. ${{|r_1|}={\sqrt{1^2+0^2+0^2}}=1}$}\\
 {Row 2 $r_2=(0, cos \phi , -sin \phi )$. ${|r_2|}={\sqrt{0^2+(\cos\phi)^2+(-\sin \phi)^2}}={\sqrt{ \cos^2 \phi + \sin^2 \phi}} = 1$}\\
 {Row 3 $r_3=(0, sin \phi , cos \phi )$. ${|r_3|}={\sqrt{0^2+(\sin\phi)^2+(\cos \phi)^2}}={\sqrt{ \sin^2 \phi + \cos^2 \phi}} = 1$}\\
 {Column 1 $c_1=(1,0,0)$. ${|c_1|}={\sqrt{1^2+0^2+0^2}}=1$}\\
 {Column 2 $c_2=(0, cos \phi , sin \phi )$. ${|c_2|}={\sqrt{0^2+(\cos\phi)^2+(\sin \phi)^2}}={\sqrt{ \cos^2 \phi + \sin^2 \phi}} = 1$}\\
 {Column 3 $c_3=(0,-sin \phi , cos \phi )$. ${|c_3|}={\sqrt{0^2+(-\sin\phi)^2+(\cos \phi)^2}}={\sqrt{ \sin^2 \phi + \cos^2 \phi}} = 1$}\\\\
Thus, since rows and columns of matrix $R_x$ are orthogonal and their length is equal to 1, we prove that they are orthonormal.

Now consider the rotation matrix around the y-axis.\\
Rotation by $\theta$ aroud $y$ axis:
\begin{equation}
	R_y = \left[
	\begin{array}{ccc}
		cos \theta & 0 & sin \theta \\
		0 & 1 & 0 \\
		-sin \theta & 0 & cos \theta
	\end{array}
	\right].
	\label{eq:kmatrixY}
\end{equation}
We can prove that the rows and columns of $R$ are orthogonal by proving the equations $R^{-1} = R^T$ and $det(R)=1.$
Let's calculate ${{R_y}^{-1}}={{(R_y^*)^T}\over{det(R_y)}}$, we need to calculate $det(R_y)$ using minors M.

${det(R_y)=(using\ second\ row)=0+1*(cos \theta * cos \theta - (-sin \theta * sin \theta)) - 0 = \cos^2 \theta + \sin^2 \theta = 1}$\\
Find transposed matrix $R_y^T$:
\begin{equation*}
	{R_y^T} = \left[
	\begin{array}{ccc}
		cos \theta & 0 & -sin \theta \\
		0 & 1 & 0 \\
		sin \theta & 0 & cos \theta
	\end{array}
	\right]
	\label{eq:kmatrixYT}
\end{equation*}
Find minor matrix M:
\begin{equation*}
	M=\left[
	\begin{array}{ccc}
		\cos\theta&0&\sin\theta\\
		0&1&0\\
		-\sin\theta&0&\cos\theta
	\end{array}
	\right]
\end{equation*}
Find adjugate matrix $R_y^{*}$:
\begin{equation*}
R_y^{*}=\left[
\begin{array}{ccc}
	cos \theta & 0 & -sin \theta \\
	0 & 1 & 0 \\
	sin \theta & 0 & cos \theta
\end{array}
\right]
\end{equation*}
Thus, we can find the value of ${{R_y}^{-1}}$
\begin{equation*}
{R_y}^{-1}={{(R_y^*)^T}\over{det(R_y)}}= {{\left[
		\begin{array}{ccc}
			cos \theta & 0 & -sin \theta \\
			0 & 1 & 0 \\
			sin \theta & 0 & cos \theta
		\end{array}
		\right]}\over {1}} =  {\left[
	\begin{array}{ccc}
	cos \theta & 0 & -sin \theta \\
	0 & 1 & 0 \\
	sin \theta & 0 & cos \theta
\end{array}
\right]} = R^T
\end{equation*} Q.E.D.\\
Now we will prove that the length of any row or column is equal to 1.\\
{Row 1 $r_1=(\cos \theta, 0, \sin \ theta)$. ${{|r_1|}={\sqrt{(\cos \theta)^2+0^2+(\sin \theta)^2}}=1}$}\\
{Row 2 $r_2=(0,1,0 )$. ${|r_2|}={\sqrt{0^2+1^2+0^2}} = 1$}\\
{Row 3 $r_3=(-\sin \theta,0 , \cos \theta )$. ${|r_3|}={\sqrt{(-\sin\theta)^2+0^2+(\cos \theta)^2}}={\sqrt{ \sin^2 \theta + \cos^2 \theta}} = 1$}\\
{Column 1 $c_1=(\cos \theta, 0, -\sin \ theta)$. ${{|c_1|}={\sqrt{(\cos \theta)^2+0^2+(-\sin \theta)^2}}={\sqrt{ \cos^2 \theta + \sin^2 \theta}}=1}$}\\
{Column 2 $c_2=(0,1,0 )$. ${|c_2|}={\sqrt{0^2+1^2+0^2}} = 1$}\\
{Column 3 $c_3=(\sin \theta,0 , \cos \theta )$. ${|c_3|}={\sqrt{(\sin\theta)^2+0^2+(\cos \theta)^2}}={\sqrt{ \sin^2 \theta + \cos^2 \theta}} = 1$}\\
Thus, since rows and columns of matrix $R_y$ are orthogonal and their length is equal to 1, we prove that they are orthonormal.

And in the end,  consider the rotation matrix around the z-axis.\\
Rotation by $\psi$ aroud $z$ axis:
\begin{equation}
	R_z = \left[
	\begin{array}{ccc}
		cos \psi & -sin \psi & 0 \\
		sin \psi & cos \psi & 0 \\
		0 & 0 & 1
	\end{array}
	\right].
	\label{eq:kmatrixZ}
\end{equation}
We can prove that the rows and columns of $R$ are orthogonal by proving the equations $R^{-1} = R^T$ and $det(R)=1.$
Let's calculate ${{R_z}^{-1}}={{(R_z^*)^T}\over{det(R_z)}}$, we need to calculate $det(R_z)$ using minors M.

${det(R_z)=(using\ third\ row)=0-0+1*(cos \psi * cos \psi - (-sin \psi * sin \psi)) = \cos^2 \psi + \sin^2 \psi = 1}$\\
Find transposed matrix $R_z^T$:
\begin{equation*}
	{R_z^T} = \left[
	\begin{array}{ccc}
		cos \psi  & sin \psi & 0  \\
		-sin \psi  & cos \psi & 0 \\
		0 & 0 & 1
	\end{array}
	\right]
	\label{eq:kmatrixZT}
\end{equation*}
Find minor matrix M:
\begin{equation*}
	M=\left[
	\begin{array}{ccc}
		cos \psi  & sin \psi & 0  \\
		-sin \psi  & cos \psi & 0 \\
		0 & 0 & 1
	\end{array}
	\right]
\end{equation*}
Find adjugate matrix $R_y^{*}$:
\begin{equation*}
	R_z^{*}=\left[
	\begin{array}{ccc}
		cos \psi  & -sin \psi & 0  \\
		sin \psi  & cos \psi & 0 \\
		0 & 0 & 1
	\end{array}
	\right]
\end{equation*}
Thus, we can find the value of ${{R_z}^{-1}}$
\begin{equation*}
{R_z}^{-1}={{(R_z^*)^T}\over{det(R_z)}}= {{\left[
		\begin{array}{ccc}
			cos \psi  & sin \psi & 0  \\
			-sin \psi  & cos \psi & 0 \\
			0 & 0 & 1
		\end{array}
		\right]}\over {1}} =  {\left[
	\begin{array}{ccc}
	cos \psi  & sin \psi & 0  \\
	-sin \psi  & cos \psi & 0 \\
	0 & 0 & 1
\end{array}
\right]} = R^T
\end{equation*} Q.E.D.\\
Now we will prove that the length of any row or column is equal to 1.\\
{Row 1 $r_1=(\cos \psi, -\sin \psi, 0)$. ${{|r_1|}={\sqrt{(\cos\psi)^2+(-\sin \psi)^2+0^2}}={\sqrt{ \cos^2 \psi + \sin^2 \psi}}=1}$}\\
{Row 2 $r_2=(\sin \psi , \cos \psi, 0 )$. ${|r_3|}={\sqrt{(\sin\psi)^2+(\cos \psi)^2 +0^2}}={\sqrt{ \sin^2 \psi + \cos^2 \psi}} = 1$\\
{Row 3 $r_3=r_2=(0,0,1 )$. ${|r_2|}={\sqrt{0^2+0^2+1^2}} = 1$}\\
{Column 1 $c_1=(\cos \psi, \sin \psi, 0)$. ${{|c_1|}={\sqrt{(\cos\psi)^2+(\sin \psi)^2+0^2}}={\sqrt{ \cos^2 \psi + \sin^2 \psi}}=1}$}\\
{Column 2 $c_2=(-\sin \psi , \cos \psi, 0 )$. ${|c_2|}={\sqrt{(-\sin\psi)^2+(\cos \psi)^2 +0^2}}={\sqrt{ \sin^2 \psi + \cos^2 \psi}} = 1$}\\
{Column 3 $c_3=(0,0,1 )$. ${|c_3|}={\sqrt{0^2+0^2+1^2}} = 1$}\\
Thus, since rows and columns of matrix $R_y$ are orthogonal and their length is equal to 1, we prove that they are orthonormal.

\textit{(b) Prove that properties $R^{-1} = R^T$ and $det(R)=1$ also hold true for $R = R_z(\psi)R_y(\theta)R_x(\phi)$}\\
Let's calculate R as the multiplication of the matrixes~\cite{Stricker2021}.: 
\begin{equation}
\begin{gathered}	
R=	 \left[
	\begin{array}{ccc}
		cos \psi & -sin \psi & 0 \\
		sin \psi & cos \psi & 0 \\
		0 & 0 & 1
	\end{array}
	\right]
	\left[
	\begin{array}{ccc}
		cos \theta & 0 & sin \theta \\
		0 & 1 & 0 \\
		-sin \theta & 0 & cos \theta
	\end{array}
	\right]
	\left[
	\begin{array}{ccc}
		1 & 0 & 0 \\
		0 & cos \phi & -sin \phi \\
		0 & sin \phi & cos \phi
	\end{array}
	\right]=\\
	\left[
	\begin{array}{ccc}
		\cos\psi\cos\theta & -\sin\psi &\cos\psi\sin\theta\\
		\cos\theta\sin\psi & \cos\psi & \sin\theta\sin\psi\\
		-\sin\theta & 0 & \cos\theta
	\end{array}
	\right]
	\left[
	\begin{array}{ccc}
		1 & 0 & 0 \\
		0 & cos \phi & -sin \phi \\
		0 & sin \phi & cos \phi
	\end{array}
	\right]=\\
	\left[
	\begin{array}{ccc}
		\cos\psi\cos\theta & -\sin\psi\cos\phi+\cos\psi\sin\theta\sin\phi & \sin\psi\sin\phi+\cos\psi\sin\theta\cos\phi\\
		\cos\theta\sin\psi & \cos\psi\cos\phi+\sin\theta\sin\psi\sin\phi  & -\cos\psi\sin\phi+\sin\theta\sin\psi\cos\phi\\
		-\sin\theta & \cos\theta\sin\phi & \cos\theta\cos\phi
	\end{array}
	\right]
	\label{eq:kmatrixZYX}
\end{gathered}	
\end{equation}
In this case, a transposed matrix $R^T$ looks like:
\begin{equation*}
	R^T=\left[
	\begin{array}{ccc}
		\cos\psi\cos\theta & \cos\theta\sin\psi & -\sin\theta\\
		-\sin\psi\cos\phi+\cos\psi\sin\theta\sin\phi & \cos\psi\cos\phi+\sin\theta\sin\psi\sin\phi & \cos\theta\sin\phi\\
		\sin\psi\sin\phi+\cos\psi\sin\theta\cos\phi & \cos\psi\sin\phi+\sin\theta\sin\psi\cos\phi & \cos\theta\cos\phi
	\end{array}
	\right]
	\label{eq:kmatrixZYXT}
\end{equation*}

Let's calculate ${{R}^{-1}}={{R^{*T}}\over{det(R)}}$, we need to calculate $det(R)$ using minors M.
 \begin{equation}
 	\begin{gathered}
 		det(R) = (using \ third \ row)= -\sin\theta M^{3}_1 - \cos\theta\sin\phi M^{3}_2 + \cos\theta\cos\phi M^{3}_3
 	\end{gathered}
 \label{eq:kmatrixDET}
\end{equation}
Calculate minors:
 \begin{equation}
	\begin{gathered}
		M^{3}_1 = 
			\left[
		\begin{array}{cc}
		 -\sin\psi\cos\phi+\cos\psi\sin\theta\sin\phi & \sin\psi\sin\phi+\cos\psi\sin\theta\cos\phi\\
		 \cos\psi\cos\phi+\sin\theta\sin\psi\sin\phi  & -\cos\psi\sin\phi+\sin\theta\sin\psi\cos\phi		 	
		\end{array}
		\right]=\\
		=(-\sin\psi\cos\phi+\cos\psi\sin\theta\sin\phi)(-\cos\psi\sin\phi+\sin\theta\sin\psi\cos\phi)-\\-(\sin\psi\sin\phi+\cos\psi\sin\theta\cos\phi)(\cos\psi\cos\phi+\sin\theta\sin\psi\sin\phi)=\\= \sin\psi \cos\phi \cos\psi \sin\phi - \sin^2\psi \cos^2\phi \sin\theta - \cos^2\psi \sin^2 \psi \sin\theta +cos\psi\sin^2\theta \sin\phi\sin\psi\cos\phi-\\-\sin\psi\sin\phi\cos\psi\cos\phi-\sin^2\psi\sin^2\phi\sin\theta-\cos^2\psi\cos^2\phi\sin\theta+\cos\psi\sin^2\theta\cos\phi\sin\psi\sin\phi=\\=-\sin\theta(\sin^2\psi\cos^2\phi+\cos^2\psi\sin^2\phi+\sin^2\psi\sin^2\phi+\cos^2\psi\cos^2\phi)=\\=-\sin\theta(\sin^2\psi(\cos^2\phi+\sin^2\phi)+\cos^2\psi(\sin^2\phi+\cos^2\phi))=-\sin\theta(\sin^2\psi+\cos^2\psi)=-\sin\theta
	\end{gathered}
\label{eq:kmatrixM31}
\end{equation}

\begin{equation}
	\begin{gathered}
		M^{3}_2 = 
		\left[
		\begin{array}{cc}
		\cos\psi\cos\theta & \sin\psi\sin\phi+\cos\psi\sin\theta\cos\phi\\
		\cos\theta\sin\psi   & -\cos\psi\sin\phi+\sin\theta\sin\psi\cos\phi\\	 	
		\end{array}
		\right]=\\
		= (\cos\psi\cos\theta)(-\cos\psi\sin\phi+\sin\theta\sin\psi\cos\phi)-(\sin\psi\sin\phi+\cos\psi\sin\theta\cos\phi)(\cos\theta\sin\psi)=\\=
		\cos\psi\cos\theta\sin\theta\sin\psi\cos\phi - \cos^2\psi\cos\theta\sin\phi - \cos\theta\sin^2\psi\sin\phi-\cos\theta\sin\psi\cos\psi\sin\theta\cos\phi=\\=
		-\cos\theta\sin\phi(\cos^2\psi+\sin^2\psi)=-\cos\theta\sin\phi
	\end{gathered}
\label{eq:kmatrixM32}
\end{equation}

\begin{equation}
	\begin{gathered}
		M^{3}_3 = 
		\left[
		\begin{array}{cc}
			\cos\psi\cos\theta & -\sin\psi\cos\phi+\cos\psi\sin\theta\sin\phi\\
			\cos\theta\sin\psi   & \cos\psi\sin\phi+\sin\theta\sin\psi\sin\phi\\	 	
		\end{array}
		\right]=\\
		= (\cos\psi\cos\theta)(\cos\psi\cos\phi+\sin\theta\sin\psi\sin\phi)-(-\sin\psi\cos\phi+\cos\psi\sin\theta\sin\phi)(\cos\theta\sin\psi)=\\=
		\cos^2\psi\cos\theta\cos\phi+\cos\psi\cos\theta\sin\theta\sin\psi\sin\phi 
		 -\cos\theta\sin\psi\cos\psi\sin\theta\sin\phi+ \cos\theta\sin^2\psi\cos\phi=\\=
		\cos\theta\cos\phi(\cos^2\psi+\sin^2\psi)=\cos\theta\cos\phi
	\end{gathered}
\label{eq:kmatrixM33}
\end{equation}

Let us substitute the obtained values of the minors ~\ref{eq:kmatrixM31},~\ref{eq:kmatrixM32} ,~\ref{eq:kmatrixM33}  into expression ~\ref{eq:kmatrixDET}:
\begin{equation}
	\begin{gathered}
		det(R) = (using \ third \ row)= -\sin\theta M^{3}_1 - \cos\theta\sin\phi M^{3}_2 + \cos\theta\cos\phi M^{3}_3 =\\=
		-\sin\theta (-\sin\theta) - \cos\theta\sin\phi (-\cos\theta\sin\phi)+ \cos\theta\cos\phi \cos\theta\cos\phi=\\=
		\sin^2\theta + \cos^2\theta\sin^2\phi + \cos^2\theta\cos^2\phi = \\=
		\sin^2\theta + \cos^2\theta(\sin^2\phi + \cos^2\phi)= \sin^2\theta + \cos^2\theta =1
	\end{gathered}
	\label{eq:kmatrixDETn}
\end{equation}
Thus, we have proved that the determinant of matrix ~\ref{eq:kmatrixZYX} is equal to 1.
Find minor matrix M:
\begin{equation*}
	M=\left[
	\begin{array}{ccc}
		\cos\psi\cos\theta & \sin\psi\cos\phi-\cos\psi\sin\theta\sin\phi & \sin\psi\sin\phi+\cos\psi\sin\theta\cos\phi\\
		-\cos\theta\sin\psi & \cos\psi\cos\phi+\sin\theta\sin\psi\sin\phi & \cos\psi\sin\phi-\sin\theta\sin\psi\cos\phi\\
		-\sin\theta & -\cos\theta\sin\phi& \cos\theta\cos\phi
	\end{array}
	\right]
\end{equation*}
Find adjugate matrix ${R^{*T}}$:
	\begin{equation*}
		R^{*T}=\left[
		\begin{array}{ccc}
			\cos\psi\cos\theta & \cos\theta\sin\psi & -\sin\theta\\
			-\sin\psi\cos\phi+\cos\psi\sin\theta\sin\phi & \cos\psi\cos\phi+\sin\theta\sin\psi\sin\phi & \cos\theta\sin\phi\\
			\sin\psi\sin\phi+\cos\psi\sin\theta\cos\phi & \cos\psi\sin\phi+\sin\theta\sin\psi\cos\phi & \cos\theta\cos\phi
		\end{array}
		\right]
	\end{equation*}
And now we can calculate ${{R}^{-1}}={{R^{*T}}\over{det(R)}}$
\begin{equation*}
	\begin{gathered}
	R^{-1}=
\frac{\left[
	\begin{array}{ccc}
		\cos\psi\cos\theta & \cos\theta\sin\psi & -\sin\theta\\
		-\sin\psi\cos\phi+\cos\psi\sin\theta\sin\phi & \cos\psi\cos\phi+\sin\theta\sin\psi\sin\phi & \cos\theta\sin\phi\\
		\sin\psi\sin\phi+\cos\psi\sin\theta\cos\phi & \cos\psi\sin\phi+\sin\theta\sin\psi\cos\phi & \cos\theta\cos\phi
	\end{array}
	\right]}{1} = \\
\left[
\begin{array}{ccc}
\cos\psi\cos\theta & \cos\theta\sin\psi & -\sin\theta\\
-\sin\psi\cos\phi+\cos\psi\sin\theta\sin\phi & \cos\psi\cos\phi+\sin\theta\sin\psi\sin\phi & \cos\theta\sin\phi\\
\sin\psi\sin\phi+\cos\psi\sin\theta\cos\phi & \cos\psi\sin\phi+\sin\theta\sin\psi\cos\phi & \cos\theta\cos\phi
\end{array}
\right]
\end{gathered}
	\label{eq:kmatrixInv}
\end{equation*}

Thus, we have proved that ${{R}^{-1}}={R^T}$
So we prove that properties $R^{-1} = R^T$ and $det(R)=1$ also hold true for $R = R_z(\psi)R_y(\theta)R_x(\phi)$ ~\cite{Stricker2021}\\
\textit{(c)Name the geometric interpretation of the determinant of a square 3 * 3 matrix? Why does a rotation matrix have to have determinant 1}\\
Let's calculate the determinant of a square 3 * 3 matrix X:
\begin{equation*}
\begin{gathered}
	det(X) = \left[
	\begin{array}{ccc}
		a_{11} & a_{12} & a_{13} \\
		a_{21} & a_{22} & a_{23} \\
		a_{31} & a_{32} & a_{33}
	\end{array}
	\right] = 	a_{11}(	a_{22}	a_{33}-	a_{23}	a_{32}) - 	a_{12}(	a_{21}	a_{33}-	a_{23}	a_{31}) + 	a_{13}(	a_{21}	a_{32}-	a_{22}	a_{31})=\\
	a_{11}a_{22}a_{33}-a_{11}a_{23}a_{32}-a_{12}a_{21}a_{33}+a_{12}a_{23}a_{31}+a_{13}a_{21}a_{32}-a_{13}a_{22}a_{31}
\end{gathered}
\label{eq:33matrix}
\end{equation*}
Now we look at the formula of the volume of parallelepiped which is used vectors:
\begin{equation*}
	\begin{gathered}
		V = |\vec{a} \vec{b}\vec{c}| = det \left[
		\begin{array}{ccc}
			a_{11} & a_{12} & a_{13} \\
			a_{21} & a_{22} & a_{23} \\
			a_{31} & a_{32} & a_{33}
		\end{array}
		\right] = 	a_{11}(	a_{22}	a_{33}-	a_{23}	a_{32}) - 	a_{12}(	a_{21}	a_{33}-	a_{23}	a_{31}) + 	a_{13}(	a_{21}	a_{32}-	a_{22}	a_{31})=\\
		a_{11}a_{22}a_{33}-a_{11}a_{23}a_{32}-a_{12}a_{21}a_{33}+a_{12}a_{23}a_{31}+a_{13}a_{21}a_{32}-a_{13}a_{22}a_{31},		
	\end{gathered}
	\label{eq:33matrix}
\end{equation*}

where $\vec{a} = (	a_{11}, a_{12}, a_{13}), \vec{b} = (	a_{21}, a_{22}, a_{23}), \vec{c} = (	a_{31}, a_{32}, a_{33})$\\
We can see what the determinant of a 3x3 square matrix is the volume of the parallelepiped built on $\vec{a},\vec{b},\vec{c}$ vectors, which is shown in Figure~\ref{fig:Pararll}.

\begin{figure}[h!]
	\centerline{\includegraphics[width=0.55\textwidth]{Parall.png}}
	\caption[Parallelepiped built on three vectors]{Parallelepiped built on three vectors.}
	\label{fig:Pararll}
\end{figure}

Since each column and row represents the coordinates of a unit vector, the length of the vectors defined by the rows and columns of the rotation matrix is 1. The determinant of the rotation matrix is +1 for a right-handed frame of reference and -1 for a left-handed one. Thus, if the determinant value is -1, the image will be inverted for a right-handed frame of reference.\\

\section{Transformation Chain}

Let's define the coordinates of the point in the camera coordinat system and in the world coordinat system(the values do not match). Let $M=(x_w,y_w,z_w)$ - point in the world coordinates and $M=(x_c,y_c,z_c)$ - point in the camera coordinates and $t= (O_c, O_w)$ is translation vector between origins. The extrinsic parameters of camera are contained in the rotation matrix R and the translation vector t, they describe the camera position. Figure~\ref{fig:WtoC} is presented for better understanding.


\begin{figure}[h!]
	\centerline{\includegraphics[width=0.55\textwidth]{WtoC.png}}
	\caption[WtoC]{W to C.}
	\label{fig:WtoC}
\end{figure}


We can represent points using this equality:
\begin{equation*}
\left(
\begin{array}{ccc}
	x_c\\
	y_c\\
	z_c
\end{array}
\right) =
t +
R\left(
\begin{array}{ccc}
	x_w\\
	y_w\\
	z_w
\end{array}
\right)
\end{equation*}

We can transform equation in the form below using homogeneous coordinates~\cite{Stricker2021}.:

\begin{equation*}
	\left(
	\begin{array}{ccc}
		x_c\\
		y_c\\
		z_c\\
		1
	\end{array}
	\right) =
	\left(
	\begin{array}{ccc}
		R & t\\
		0_{3}^T & 1
	\end{array}
	\right)
	\left(
	\begin{array}{ccc}
		x_w\\
		y_w\\
		z_w\\
		1
	\end{array}
	\right) 
\end{equation*}
Now we can get a point in the 3D camera coordinate system, project the point onto the image plane and calculate its position on the image. (Figure~\ref{fig:WtoC})\\
We represent image point $(x_i,y_i)$:\\
$x_i=f{{x_c}\over{z_c}}$\\
$y_i=f{{y_c}\over{z_c}}$\\
We can present this equations in matrix form:
\begin{equation*}
\left(
\begin{array}{ccc}
	x'\\
	y'\\
	z'
\end{array}
\right) =
\left(
\begin{array}{ccc}
	f & 0 & 0\\
	0 & f & 0\\
	0 & 0& 1
\end{array}
\right) 
\left(
\begin{array}{ccc}
	x_c\\
	y_c\\
	z_c
\end{array}
\right) 
\end{equation*}
In this case, perspective projection becomes linear in homogeneous coordinates:
\begin{equation*}
	\left(
	\begin{array}{ccc}
		u\\
		v\\
		w
	\end{array}
	\right) =
	\left(
	\begin{array}{cccc}
		f & 0 & 0 & 0\\
		0 & f & 0 & 0\\
		0 & 0 & 1 & 0
	\end{array}
	\right) 
	\left(
	\begin{array}{ccc}
		x_c\\
		y_c\\
		z_c\\
		1
	\end{array}
	\right) 
\end{equation*}

Obviously, the optical center (principal point)  $(x_0, y_0)$ of the camera may not matched with the center of the image coordinate system, and there may also be a skew. We use principal point $(x_0, y_0)$, s - skew parameter and $\alpha_x = fk_x$, $\alpha_y = fk_y$ - scaling factors, so we can get the calibration matrix K, its contains the intrinsic camera parameters:
\begin{equation*}
	K=
	\left(
	\begin{array}{ccc}
		\alpha_x & s & x_0 \\
		0 & \alpha_y & y_0 \\
		0 & 0 & 1 
	\end{array}
	\right)  
\end{equation*}\\
 We can represent formula:
\begin{equation*}
	\left(
	\begin{array}{ccc}
		u'\\
		v'\\
		w'
	\end{array}
	\right) =
	\left(
	\begin{array}{cccc}
		\alpha_x & s & x_0 & 0\\
		0 & \alpha_y & y_0 & 0\\
		0 & 0 & 1 & 0
	\end{array}
	\right) 
	\left(
	\begin{array}{ccc}
		x_c\\
		y_c\\
		z_c\\
		1
	\end{array}
	\right) = K(I_3|0_3)	\left(
	\begin{array}{ccc}
		x_c\\
		y_c\\
		z_c\\
		1
	\end{array}
	\right) 
\end{equation*}

So we can find the pixel coordinates (u, v):
$x_{pix}={{u'}\over{w'}}; \ y_{pix}={{v'}\over{w'}}$

\section{Implementation}

Results of program work:
- projected points without correction of the distortion in red

\begin{figure}[h!]
	\centerline{\includegraphics[width=0.55\textwidth]{noCorr.png}}
	\caption[wCorr]{Projected points without correction of the distortion}
	\label{fig:wCorr}
\end{figure}
\newpage
- projected points with correction of the radial distortion (k1, k2 and k5) in green
\begin{figure}[h!]
	\centerline{\includegraphics[width=0.55\textwidth]{withCorr.png}}
	\caption[Corr]{ Projected points with correction of the radial distortion (k1, k2 and k5)}
	\label{fig:Corr}
\end{figure}




\bibliographystyle{plain}
\bibliography{bibliography.bib}
\end{document}